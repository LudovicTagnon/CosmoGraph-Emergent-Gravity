% CosmoGraph: Emergent Gravity from Information Geometry (Article 1 - Cosmo/Topo)
\documentclass[11pt]{article}
\usepackage[a4paper,margin=1in]{geometry}
\usepackage{amsmath,amssymb}
\usepackage{graphicx}
\usepackage{hyperref}
\usepackage{booktabs}

\graphicspath{{python_project/outputs/}}

\title{Graph Topology as a Proxy for Large-Scale Clustering\\(An Information-Geometry Perspective)}
\author{CosmoGraph Collaboration}
\date{\today}

\begin{document}
\maketitle

\begin{abstract}
We test to what extent graph topology can serve as a complementary proxy for the clustering usually attributed to the underlying mass (including dark matter). Galaxies are nodes, proximity links define the graph, and node centrality plays the role of an information-based ``mass'' proxy. Using Sloan Digital Sky Survey (SDSS) data, we measure: (i) the correlation between local strength (weighted degree) and centralities; (ii) the power spectrum $P(k)$ of the galaxy distribution versus the power spectrum obtained when weighting nodes by centrality; and (iii) a transfer function $T(k)=P_{\mathrm{real}}/P_{\mathrm{random}}$ to remove grid artifacts. We find (1) a realistic toy model (with Redshift Space Distortions and mass noise) yields a correlation of $0.259$, close to SDSS ($0.223$), while the ideal model reaches $0.397$; (2) raw spectral slopes are $-2.98$ (mass) and $-2.72$ (topology) with a bias $\simeq 1.5$; (3) transfer-function slopes are $-1.61$ (mass) and $-1.32$ (topology), showing centrality captures a non-trivial fraction of the clustering signal beyond voxel artifacts. Randomized catalogs collapse both mass and topology to the grid slope ($\sim-1.4$), confirming the signal is cosmological. We provide bootstrap errors on node-level correlations; uncertainties on power-spectrum slopes are left for future work. The results should be viewed as a topological proxy, not a replacement for standard mass-based analyses.
\end{abstract}

\section{Introduction}
The large-scale universe exhibits a scale-free ``cosmic web''. We explore whether part of the phenomenology usually attributed to dark matter can be captured by the topology of a galaxy network: galaxies are nodes; edges connect near neighbors; centrality (information flow) stands in for mass. Rather than assuming a specific dark-matter particle model, we test how much clustering signal is encoded in the graph.

\section{Data and Graph Construction}
\paragraph{SDSS slice.} We select galaxies with $0.04<z<0.12$, $130<\mathrm{RA}<240$, $-5<\mathrm{DEC}<60$, then restrict to an inner buffer ($0.05<z<0.11$, $135<\mathrm{RA}<235$, $0<\mathrm{DEC}<55$) for statistics. Positions are converted to Cartesian $(x,y,z)$ using the low-$z$ approximation $D \simeq cz/H_0$ (cosmology not critical for this local slice). We build a $k$-NN graph with $k=10$, edge weights $w=1/d$, and keep the largest connected component (LCC), containing $\sim$3.5k galaxies in the buffered slice (robustness run shown on a 5k subsample, LCC $\sim$1909 for speed). Strength $S_i=\sum_j w_{ij}$ (``mass proxy'') and eigenvector centrality $C_i$ (information proxy) are computed on the LCC.

\paragraph{Toy universes.} An ``ideal'' toy universe uses synthetic clusters/filaments (3\,000 points, $k=20$). A ``realistic'' toy adds Redshift Space Distortions (Gaussian noise on $z$, $\sigma_z\approx 5\%$ of box size) and lognormal scatter on mass (median 1, $\sigma_{\ln M}=0.5$).

\section{Spectral Analysis}
We voxelize the point set on a $64^3$ grid (box size $\sim$300\,Mpc units), form overdensity $\delta=(\rho-\bar\rho)/\bar\rho$, and compute $P(k)$ via FFT. Grid artifacts are removed by dividing by a randomized catalog: $T(k)=P_{\mathrm{real}}(k)/P_{\mathrm{rand}}(k)$. We compare mass weights ($w=1$) to topology weights ($w\propto C_i$).

\section{Results}
\subsection{Correlations}
Pearson correlations (Strength vs Centrality), bootstrap errors (10k resamples):
\begin{itemize}
  \item Ideal toy: $0.397$
  \item Realistic toy (RSD + noise): $0.259$
  \item SDSS (buffered LCC): $0.223$
  \item SDSS robustness (5k subsample, LCC $\sim$1909):
    \begin{itemize}
      \item eigenvector: $k=5 \Rightarrow 0.20\pm0.05$, $k=10 \Rightarrow 0.16\pm0.05$, $k=20 \Rightarrow 0.20\pm0.04$;
      \item closeness: $k=5 \Rightarrow 0.07\pm0.02$, $k=10 \Rightarrow 0.09\pm0.02$, $k=20 \Rightarrow 0.19\pm0.02$;
      \item degree: $1.0$ (identique à la strength).
    \end{itemize}
    Skewness of eigenvector centrality $\sim 26$ (few hubs dominate).
\end{itemize}
The realistic toy converges toward the observed SDSS value once observational distortions are applied. 

\subsection{Power spectra}
Raw $P(k)$ slopes: $-2.98$ (mass) vs $-2.72$ (topology); bias $\sim 1.5$. Random catalogs show grid-induced slopes $\sim -1.4$ (no cosmological signal).

Transfer-function slopes (signal cleaned of grid effects): $-1.61$ (mass) vs $-1.32$ (topology). A jackknife on 8 RA/DEC/$z$ quantile sub-volumes (recomputing LCC, weights, and randoms each time) yields $P$: $-2.47\pm0.31$ (mass), $-2.13\pm0.38$ (topo); $T$: $-1.15\pm0.60$ (mass), $-0.80\pm0.71$ (topo). Errors are large but consistent with the point estimates and confirm that topology tracks most of the clustering signal beyond voxel artifacts.

\subsection{Structure in configuration space}
In X--Z projection, the realistic toy (with RSD) visually matches SDSS ``Fingers of God'' elongations, while the ideal toy remains sharper. This supports the interpretation that observational effects explain much of the gap between theory and data.

\section{Figures}
\begin{figure}[h]
  \centering
  \includegraphics[width=0.85\textwidth]{final_theory_vs_reality.png}
  \caption{Correlation benchmark. Bars: ideal toy (0.397), realistic toy (0.259), SDSS (0.223).}
\end{figure}

\begin{figure}[h]
  \centering
  \includegraphics[width=0.95\textwidth]{power_spectrum_combined.png}
  \caption{Left: raw $P(k)$ (mass vs topology) with bias. Right: transfer functions $T(k)=P_{\mathrm{real}}/P_{\mathrm{random}}$; slopes $-1.61$ (mass) and $-1.32$ (topology).}
\end{figure}

\begin{figure}[h]
  \centering
  \includegraphics[width=0.7\textwidth]{power_with_lcdm.png}
  \caption{$P(k)$ (mass/topology) with a $\Lambda$CDM reference table ($k, P(k)$ tabulated); BAO/wiggles not resolved at this resolution. Error bars on $P(k)$/$T(k)$ to be added in future jackknife/bootstrap.}
\end{figure}

\begin{figure}[h]
  \centering
  \includegraphics[width=0.48\textwidth]{sensitivity_corr_vs_k.png}
  \includegraphics[width=0.48\textwidth]{centrality_hist.png}
  \caption{Left: robustness of corr(strength, centrality) vs $k$ (bootstrap, SDSS subsample). Right: eigenvector centrality distribution (highly skewed; few hubs dominate).}
\end{figure}

\begin{figure}[h]
  \centering
  \includegraphics[width=0.95\textwidth]{structural_comparison_2d_XZ.png}
  \caption{X--Z projection. Left: ideal toy (sharp filaments). Middle: realistic toy (RSD + noise, elongated clusters). Right: SDSS slice (observed elongations).}
\end{figure}

\section{Discussion}
The clustering excess beyond random is encoded similarly in mass and topological centrality. The realistic toy bridges most of the gap to SDSS once RSD and mass scatter are included, suggesting that observational effects explain much of the reduced correlation. Limitations: moderate correlation ($\sim$0.22) on SDSS, sensitivity to $k$-NN choice and slice, centrality values are highly skewed (a few hubs dominate), no explicit baryon/fermion sector, and only a preliminary jackknife on $P(k)$/$T(k)$ (large errors; full error budget deferred). Scripts for k-scan/centrality variants and corr error estimates are provided; a full error budget and higher-resolution $\Lambda$CDM comparison (with BAO) are deferred. Future work: larger surveys (BOSS/eBOSS), error bars via $k$ and slice variations plus bootstrap/jackknife, and analytical links between centrality and density contrast.

\section{Conclusion}
We provide a falsifiable proposal that a non-trivial part of the clustering signal usually attributed to dark matter can be encoded in the topology of the galaxy network. Centrality reproduces the cosmological clustering signal in both configuration space (Fingers of God) and Fourier space (transfer-function slopes near mass). This opens a data-driven path to ``information gravity'' as a complementary description to particle dark matter.

\end{document}
