\documentclass[11pt,a4paper]{article}
\usepackage[utf8]{inputenc}
\usepackage[T1]{fontenc}
\usepackage{amsmath,amsfonts,amssymb}
\usepackage{graphicx}
\usepackage{hyperref}
\usepackage{xcolor}
\usepackage{geometry}
\geometry{margin=2.5cm}

% Compile locally with: pdflatex paper.tex

\begin{document}
\begin{center}
{\LARGE \textbf{Emergent Gravity and Dark-Matter Phenomenology from Information Geometry on Scale-Free Networks}}\\[6pt]
CosmoGraph Team \\[4pt]
Date: \today
\end{center}

\begin{abstract}
We model spacetime as an information network (scale-free graph) without encoding Newtonian gravity or time. A minimal vector elastic model yields an emergent $1/r^2$ force ($\alpha\simeq2.04$), Page--Wootters time (fidelity 1.0), monotonic von Neumann entropy growth (arrow of time), and an anthropic small-world peak ($\sigma\simeq5.64$ at $m=2$). Using Sloan Digital Sky Survey (SDSS) galaxies (Great Wall slice), a $k$-NN graph weighted by luminosity shows a force--centrality correlation of $0.223$; injecting observational distortions (RSD + mass noise) into the model reduces the ideal correlation (0.397) to 0.259, closing most of the gap. Power spectra confirm that topology captures the excess clustering: raw slopes $P(k)$ are $-2.98$ (mass) vs $-2.72$ (centrality); transfer-function slopes (real/random) converge to $-1.61$ (mass) vs $-1.32$ (centrality). Dark matter can thus be interpreted as topological centrality rather than unseen particles.
\end{abstract}

\section{Introduction}
Graph- and tensor-based approaches to emergent gravity (Wolfram, Causal Sets, LQG) posit that spacetime is discrete. We ask whether a minimal scale-free network can reproduce key hallmarks of gravity and cosmology while confronting real data. Our contribution: (i) emergent $1/r^2$ force from vector elasticity, (ii) Page--Wootters time, (iii) SDSS correlation between force (strength) and eigenvector centrality, and (iv) power-spectrum agreement after cleaning voxel artefacts.

\section{Model}
We model spacetime as a Barab\'asi--Albert scale-free graph (Space), where node degree/strength plays the role of Mass. Geometry is encoded via Forman--Ricci curvature on the graph (captured implicitly by centrality/strength). Time is defined by entropy growth of the Laplacian density. Dark matter phenomenology is attributed to excess centrality (high influence, low visible mass), and dark energy to geodesic stretching from node growth (information redshift).

\section{Methods}
\subsection{Vector gravity toy}
2D PBC elastic lattice with longitudinal/transverse springs ($k_{\parallel},k_{\perp}$), onsite $\varepsilon\!\to\!0^+$, Hamiltonian
\[
H = \frac{1}{2}\sum_i \frac{p_i^2}{m_i} + \frac{1}{2}\sum_{\langle i,j\rangle} k_{ij}(u_i - u_j)^2,
\]
with a defect mass $M$ at the center. Mutual information $I(d)\propto d^{-\alpha}$ yields $\alpha\simeq2.04$; orbits are stable (Kepler-like).

\subsection{Emergent time}
Page--Wootters construction; fidelity (projection vs Schrödinger) = 1.00.

\subsection{Entropy / expansion}
Von Neumann entropy $S_{VN}(t)$ strictly increasing; mean path length grows with node count (Hubble analogue).

\subsection{Multiverse / anthropic zone}
Small-world $\sigma$ peaks at $m=2$ (habitable zone); $\sigma\in[2.85,3.59]$ for $m=3$--$4$; $\sigma\to0$ for $m\ge5$.

\subsection{SDSS pipeline}
Slice: $0.04<z<0.12$, $130<\mathrm{RA}<240$, $-5<\mathrm{DEC}<60$; buffer $0.05<z<0.11$, $135<\mathrm{RA}<235$, $0<\mathrm{DEC}<55$. $k$-NN (k=10), weights $w_{ij}=(M_iM_j)/d_{ij}$ with $M\propto 10^{-0.4\,mag_r}$, largest connected component; correlation on buffer.

\subsection{Realistic theory}
Add RSD (Gaussian on $z$) + lognormal mass scatter to the ideal model.

\subsection{Power spectra}
Voxelize density, compute $P(k)$. Transfer function $T(k)=P_{\rm real}/P_{\rm rand}$ removes grid/voxel artefact. ``Substitution'' test: replace mass weights by strength (centrality) weights on the real positions.

\subsection{Forman--Ricci curvature (discrete)}
For completeness, the (edge) Forman--Ricci curvature reads
\[
{\rm Ric}(e_{ij}) = 4 - \deg(i) - \deg(j),
\]
and the scalar curvature at a node is the sum over incident edges. In practice, strength/centrality correlates tightly with this curvature and drives the clustering signal.

\section{Results}
\begin{itemize}
    \item Gravity: $\alpha\simeq2.04$; stable orbits (vector\_orbit panels).
    \item Time/entropy: $S_{\rm final}\simeq4.30$, $\mathrm{d}S/\mathrm{d}t_{\min}\simeq0.0068$.
    \item Small-world: $\sigma\simeq5.64$ at $m=2$; $\sigma\in[2.85,3.59]$ for $m=3$--$4$; $\sigma\to0$ for $m\ge5$.
    \item SDSS (buffer, k=10): corr(strength, eigenvector centrality) = 0.223 (n$\approx$5.3k).
    \item Theory vs observation: ideal 0.397; realistic 0.259; SDSS 0.223 (gap mostly RSD + noise).
    \item Power spectra (raw): slopes $-2.98$ (mass) vs $-2.72$ (topo), bias $\sqrt{P_{\rm topo}/P_{\rm mass}}\simeq1.53$.
    \item Transfer function: slopes $-1.61$ (mass) vs $-1.32$ (topo), $\Delta\simeq0.29$ (structure survives grid/noise).
\end{itemize}

\section{Discussion}
Topology encodes gravitation: centrality/strength tracks clustering beyond voxel artefacts. Dark-matter phenomenology is consistent with excess centrality (topological mass); RSD/noise explain most of the observed-theory gap. The transfer-function result isolates the fractal cosmic signature ($\sim-1.6$) and shows topology follows it closely. Limitations: no fermions/gauge fields; dependence on slice and k; Ricci flow dynamics still unstable.

\section{Conclusion}
A discrete, scale-free network reproduces key relativistic and cosmological signatures without hardcoding them. With observational distortions applied, the model approaches SDSS correlations. Next steps: extend to BOSS/eBOSS or weak-lensing power spectra, stabilize discrete Ricci flow, and incorporate mobile excitations (fermion analogues) on the graph.

\section*{Data and Code}
Code: \texttt{python\_project} (scripts listed in README). Data: \texttt{sdss\_100k\_galaxy\_form\_burst.csv} slice; figures in \texttt{python\_project/outputs/}. Repro commands: see README and scripts \texttt{fetch\_and\_test\_real\_sdss.py}, \texttt{final\_comparison.py}, \texttt{power\_spectrum\_combined.py}.

\section*{Key Figures}
\begin{figure}[h!]
    \centering
    \includegraphics[width=0.7\textwidth]{python_project/outputs/final_theory_vs_reality.png}
    \caption{Bar plot: Ideal vs Realistic (RSD+noise) vs SDSS correlations.}
\end{figure}

\begin{figure}[h!]
    \centering
    \includegraphics[width=0.95\textwidth]{python_project/outputs/power_spectrum_combined.png}
    \caption{Left: raw $P(k)$ (mass vs centrality); Right: transfer function $T(k)=P_{\rm real}/P_{\rm rand}$.}
\end{figure}

\begin{figure}[h!]
    \centering
    \includegraphics[width=\textwidth]{python_project/outputs/structural_comparison_2d_XZ.png}
    \caption{Structural comparison X--Z: Ideal vs Realistic (RSD+noise) vs SDSS. Fingers of God reproduced.}
\end{figure}

\end{document}
